\chapter{Opis projektnog zadatka „PLESNJACI“ }
\section{Uvod}
		
		Cilj ovog projektnog zadatka je razviti programsku podršku za izradu web aplikacije koja će olakšati pronalaženje i organizaciju plesnih tečajeva i plesnjaka. U nastavku se opisuju najopćenitije stranice web-aplikacije i njihove funkcionalnosti. 
		
\section{Opis funkcionalnosti web aplikacije}
		 
		      Gradimo jednostavnu CRUD aplikaciju, u nastavku navodimo njene osnovne cjeline i prikaz funkcionalnosti.
		 
	 	\subsection{"Početna stranica"}
	
			Prilikom pokretanja web-aplikacije prikazuje se „Početna stranica“. Ako korisnik nije       registriran na početnoj stranici se prikazuju profili klubova i tipovi plesa koje taj klub nudi. Također na stranici se nalazi karta s dostupnim plesnjacima i lokacijama klubova. Neregistriranom korisniku je omogućeno kreiranje računa (registracija) . 
		

			 
			\subsubsection{Karta}
				
			Na karti se mogu filtrirati plesnjaci i klubovi. Plesnjaci se mogu filtrirati po tipu plesa, klubu koji ga organizira i tipovima plesa. Klubovi se mogu filtrirati po tipovima plesa za koje organiziraju tečajeve.
			\bigskip
			\bigskip
			\bigskip
			\bigskip
			\bigskip
			
			
				
			\subsubsection{Registracija}
				
			Ako se korisnik odluči registrirati otvara se odgovarajući obrazac za registraciju koji od njega zahtijeva da upiše sljedeće informacije :
				
				\begin{packed_item}
					\item Korisničko ime
					\item Lozinka
					\item Ime
					\item Prezime
					\item Spol
					\item Datum rođenja
					\item Broj mobitela
					\item Email adresa
					\item Opis plesnog iskustva i fotografija (opcionalno)
				\end{packed_item}
			\bigskip
			
	
  	\subsection {"Stranica u kojoj je korisnik registriran"}
  	Kad se korisnik registrirao on ima ulogu klijenta. Klijent može dodatno stvoriti klub. Ako klijent odluči stvoriti novi klub otvara se odgovarajući obrazac koji zahtijeva sljedeće informacije za taj klub :
		
		
		\begin{packed_item}
			\item Korisničko ime
			\item Lozinka
			\item Ime kluba
			\item Adresa sjedišta
			\item Datum rođenja
			\item Telefon
			\item Email adresa
		\end{packed_item}
	
	Novo registrirani klubovi prvo moraju biti potvrđeni od strane administratora. Ako je korisnik uspio registrirati svoj klub on postaje vlasnikom kluba. Osim uloge klijenta i vlasnika kluba postoji i uloga administratora sustava.
	
	\bigskip
	\bigskip
	\bigskip
	\bigskip
	\bigskip
	\bigskip
	\bigskip
	
		
		
		
		\subsubsection{Registrirani korisnik (klijent)}
	
			
		Klijent može pregledati, mijenjati osobne podatke i izbrisati svoj korisnički račun. Također na karti mu je omogućeno prikazivanje tečajeva koji su slobodni za upis, a rezultate može filtrirati po željenom vremenu i tipu plesa. 
		Odabirom tečaja, otvaraju se relevantne informacije poput vrste plesa, kalendar te ime, prezime i slika trenera. Kalendar ima zapisane termine tečaja i lokacije s dvoranom za svaki termin.
		

			
			\subsubsection{Registrirani korisnik(vlasnik kluba)}
			Vlasnik kluba je zadužen za organizaciju plesnjaka koji se mogu izvoditi i na lokacijama izvan prostorija kluba, a sadrže naziv, opis i sliku. Na plesnjaku se mogu plesati različiti tipovi plesa. Svaki klub ima svoju naslovnu stranicu (profil) i popis trenera.
			Svaki trener ima podstranicu na kojima ima popis grupa koje trenira, a svi termini tečaja koje vodi su mu prikazani na kalendaru. Trener prilikom otvaranja tečaja za kojeg je zadužen uz opće informacije dobiva popis klijenata koji bi trebali biti nazočni.
			Na profilu kluba su prikazane informacije o :
			
			
			
			\begin{packed_item}
				\item imenu kluba
				\item kontakt telefon
				\item email adresa
				\item kratki opis
				\item poveznica na stranicu s grupama za upis
				\item popis plesova koje njihovi treneri mogu voditi
				\item prikaz lokacija na karti
				\item popis dvorana po lokacijama
			\end{packed_item}
		
		Vlasnik kluba na naslovnoj stranici kluba može objaviti upise za tečaj s krajnjim rokom prijave. Upisi u grupu mogu biti ograničavajući ovisno o dobi i spolu. Grupi je dodijeljen trener iz kluba, skup treninga kroz neko vrijeme, maksimalni broj sudionika te opis u kojem pišu dodatne informacije o težini treninga, posebnim uvjetima treniranja, pravila ponašanja i slično.
		
			\bigskip
			\bigskip
			\bigskip
			\bigskip
			\bigskip
			
			\subsubsection{Poveznica klijenta i klijenta kao vlasnika kluba}
			
		Bilo koji klijent može vlasniku kluba poslati prijavu da postane trener, a vlasnik kluba ga potom može  potvrditi. Prijava sadrži:
		
			\begin{packed_item}
				\item motivacijsko pismo
				\item potvrda da je klijent osposobljen držati tečaj plesa (pdf dokument)
			\end{packed_item}
		Klijent može pregledati aktivne prijave na tečaj od kluba i prijaviti se. Nakon isteka roka upisa, vlasnik kluba treba odabrati klijente koji se primaju u grupu. Vlasnik kluba može naknadno mijenjati popis klijenata, uređivati i brisati grupe.
		
			\bigskip
			
			\subsubsection{Administrator}
			
			Administrator može dodati, mijenjati i brisati plesove. Odobrava novo registrirane klubove kako bi postali valjani. Također može pregledati popis svih klijenata i klubova te uređivati korisničke račune.
			Plesovi sadrže:
			\begin{packed_item}
				\item naziv
				\item kratki opis
				\item sliku
				\item link na video primjer tog plesa
			\end{packed_item}